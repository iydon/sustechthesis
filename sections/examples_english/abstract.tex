% !Mode:: "TeX:UTF-8"
% !TEX program  = xelatex

\begin{英文摘要}{Keyword A; Keyword B; Keyword C}
  Most of the thesis templates I have encountered are in the form of document classes, with a few as macro packages. Moreover, these templates often include various examples (except for those maintained frequently), resulting in template files cluttered with content unrelated to formatting. The codebase is massive, documentation is lacking, and modifications are difficult. When issues arise, one often needs to delve into the source code of the macro package or document class. Therefore, adhering to the principle of providing only the most basic necessary features, I refactored the previously written \TeX\ template. Since this is the second year of using this template, I cannot guarantee that the template interface will remain unchanged in the future—everything is subject to the documentation. After all, the university is in its early stages of development, and all documents are gradually being refined. In previous years, the school's logo and name underwent changes, and the thesis formatting also saw significant revisions. However, I can assure you that the template's interfaces are all in Chinese\footnote{Leveraging \hologo{XeLaTeX} features enhances recognizability while avoiding strict adherence to English naming conventions. Of course, this approach has some drawbacks, which I won’t elaborate on here.}, and it will be updated at least until 2021, the year of my graduation.
  
  A template cannot be expected to work perfectly forever. On one hand, the university's standards and regulations are constantly evolving; on the other hand, \hologo{LaTeX} macro packages are also changing. Popular macro packages from earlier years may become "obsolete" a few years later. Therefore, your usage and feedback are the driving forces behind my continuous updates. I sincerely welcome any suggestions and advice.
  \end{英文摘要}
  
  \begin{中文摘要}{关键词甲;关键词乙;关键词丙}
    笔者见到的毕业论文模板,大多是以文类的形式,少部分以宏包的形式,并且在模板中大多掺杂着各式各样的例子(除了维护频率高的模板),导致模板文件使用了大部分与形式格式不相关的内容,代码量巨大文档欠缺且不容易修改,出现问题需要查看宏包或者文类的源代码。于是,秉着仅提供实现最基本要求的理念,重构了之前所写的 \TeX\ 形式。由于第二年使用该模板,所以设计出的模板接口不能保证以后不发生重大变动,一切以文档为主。毕竟学校在发展初期,各类文件都在日渐完善,前几年时,学校标志及名称还发生变化,同时毕业论文的样式也发生了重大变化。但是可以保证的是,模板提供的接口均为中文形式\footnote{使用 \hologo{XeLaTeX} 特性,一方面增加辨识度,另一方面不拘泥于英文命名的规则。当然此举也有些许弊端,在此就不过多展开。},并且至少更新到 2021 年,也就是笔者毕业。模板这种东西不能保证一劳永逸,一方面学校的标准制度都在发生着改变,另一方面 \hologo{LaTeX} 的宏包也在发生着改变,早先流行的宏包可能几年后就被“淘汰”掉。因此,您的使用与反馈是我不断更新的动力,希望各位不吝赐教。
  \end{中文摘要}

  \cleardoublepage

